\documentclass[11pt,a4paper]{article}
\usepackage{siunitx} % si eenheden
\usepackage{hyperref}					% maak PDF van de thesis navigeerbaar
\usepackage{url}
\usepackage[toc,acronym,xindy]{glossaries}%voor afkortingen
\usepackage{amsmath}
\usepackage[official]{eurosym} % om euro symbool te kunnen gebruiken
\usepackage{graphicx}
\usepackage{svg}
\usepackage[final]{pdfpages}

% =========== BELANGRIJK VR IN HET VERSLAG: Tonen dat je hebt nagedacht over dimensionering componenten! ===========

% MFA: zet zoekpad voor figure
\graphicspath{{fig/}}
\usepackage{float}                      % De optie H voor de plaatsing van figuren op de plaats waar je ze invoegt. bvb. \begin{figure}[H]

\begin{document}
\title{Embedded systems 2\\
	\Huge Report Cosy Cafeteria
}
\author{Robin Van de Poel\and Arthur Van den Storme\and Tobias Cromheecke\and Thomas Feys}
\date{\today}
\maketitle
\newpage

\tableofcontents
\newpage

\section{Intro}
To improve the experience of the students in the cafeteria, an embedded system will be developed to gather a variety of data. This data will then be display to the students through an online medium such as a website. The main focus will be to gather information about the amount off people that are in the cafeteria. There are two main goals. The first goal is to display the estimated waiting time to get a meal to the students. Secondly the amount of available seats will be tracked and displayed to the students. Next to this some additional information will be gathered. The temperature, air-quality and sound level will be monitored to give the student a general idea about the conditions in the cafeteria. This way the student can gauge if the cafeteria is suitable to study in at a certain moment. An overview of the envisioned system is displayed in figure~\ref{fig:system}. Al the developed hardware and software is displayed at: \url{https://github.com/thomasf10/cosy_cafeteria}.
\begin{figure}[!ht]
	\centering
	\includegraphics[width=1.0\linewidth]{situation.pdf}
	\caption{Situation sketch}
	\label{fig:system}
\end{figure}


\section{Functionality}
%- wat gaat uw systeem juist doen

The developed system uses three sensors: 
\begin{itemize}
	\item AMG8833: IR grid sensor
	\item CSS811: air quality sensors
	\item Sparkfun sound detector: microphone
\end{itemize}
By incorporating these sensors in to the system, a variety of sensor data is gathered, namely: ambient temperature [\SI{}{\celsius}], $CO_2$-level [ppm], TVOC-level (total volatile organic compounds) [ppb] and sound level [db]. Next to this the AMG8833 captures the temperature af a two dimensional area in 8 by 8 pixels. \\

The 'brain' of the system is an ESP32, this microcontroller has an integrated Wi-Fi module, which will be used to transmit the data. The gathered data is sent to a server, which stores the received data in a database. \\

Alongside the reception of data from the ESP32, the server hosts a website which is used to display the data to the students. The website uses the pixel data to generate a heatmap of the cafeteria, this allows the students to see how many people are present in the cafeteria. Unfortunately one module cannot cover the entire cafeteria, to generate the heatmap. In order to do this, several modules have to be deployed in the cafeteria. In this project only one module will be constructed as a proof of concept. Next to the heatmap, the other sensor data is displayed on the website this includes: ambient temperature, $CO_2$-level, TVOC-level and the sound-level. Based on these parameters the students can gauge if the cafeteria is currently suitable as a study-location.

\section{System architecture}
systeem wat uitleggen \dots
\textbf{EENS CHECKEN OF DE ARCHITECTUUR IN GROTE LIJNEN KLOPT}
An overview of the full system is displayed in figure~\ref{fig:architecture}.
\begin{figure}[!ht]
	\centering
	\includegraphics[width=1.0\linewidth]{architecture.pdf}
	\caption{System architecture}
	\label{fig:architecture}
\end{figure}


\section{Power management}
\subsection{Battery}
Specifications:
\begin{itemize}
	\item Rated capacity: 3200 mAh
	\item Nominal voltage: 3.7 V
	\item Maximum voltage: 4.2 V
	\item Maximum charge current: 1625 mA
	\item Maximum discharge current: 6400 mA\footnote{Maximum discharge rate is 2C. This means discharging the battery capacity in 0.5 hours $\frac{3200 mAh}{0.5 h} = 6400 mA$ .}
	\item Maximum discharge voltage: 2.5 V
	\item Charge time: 4 hours
	\item No protection circuit
\end{itemize}
\subsection{Battery charging circuit}
\subsection{Battry protection circuit}
\subsection{Polarity protection}
\section{Microcontroller}
\section{Sensorboard}

% ============ Geofrey wil zien dat we hebben nagedacht over de dimensionering van de componenten !!! ============


\section{Selection battery charger components}
There are many manufacturers who produce charging IC's  bla bla bla ...

We considerd many chips from different kind of manufacturers. Eventualy we went with the BQ24075 IC from Texas Instruments. It's a great chose because it's made to charge a single cell Li-Ion battery (that's all we need), and it's perfectly fit to charge using a USB port because the IC has a selectable 100 mA and 500 mA maximum input current \textbf{TO DO: VERWIJZEN NAAR USB TABEL}. \textbf{OOK NOG ZEGGEN DAT DE OPLAADSTROOM OVEREENKOMT MET DIE VAN DE BATTERY}. In the subsections below you'll find the calculations.

\subsection{Charging}
Set $\overline{CE}$ low to initiate battery charging. The battery is charged in three phases: 
\begin{enumerate}
	\item Conditioning pre-charge
	\item Constant current fast charge (current regulation) 
	\item Constant voltage tapering (voltage regulation)
\end{enumerate}
\begin{figure}[!ht]
	\centering
	\includegraphics[width=1.0\linewidth]{Charge_cycle}
	\caption{Charge cycle}
	\label{fig:Charge _cycle}
\end{figure}
In the pre-charge phase, the battery is charged at with the pre-charge current $I_{PRECHG}$. Once the battery voltage crosses the $V_{LOWV}$ threshold, the battery is charged with the fast-charge current $I_{CHG}$. As the battery voltage reaches $V_{BAT(REG)}$, the battery is held at a constant voltage of $V_{BAT(REG)}$ and the charge current tapers off as the battery approaches full charge. When the battery current reaches $I_{TERM}$, the CHG pin indicates charging done by going high-impedance.

The value of the fast-charge current is set by the resistor connected from the ISET pin to VSS, and is given by the equation:
\[ I_{CHG} = \frac{K_{ISET}}{R_{ISET}} \]
The charge current limit is adjustable up to 1.5 A. The valid resistor range is 590 $\Omega$ to 8.9 k$\Omega$. If $I_{CHG}$ is programmed as greater than the input current limit, the battery will not charge at the rate of $I_{CHG}$, but at the slower rate of $I_{IN(MAX)}$ (minus the load current on the OUT pin, if any). In this case, the charger timers will be proportionately slowed down.

\subsubsection{Charge Current Translator}
When the charger is enabled, internal circuits generate a current proportional to the charge current at the ISET input. The current out of $I_{SET}$ is 1/400 ($\pm$10\%) of the charge current. This current, when applied to the external charge current programming resistor, $R_{ISET}$, generates an analog voltage that can be monitored by an external host to calculate the current sourced from BAT.
\[ V_{ISET} = \frac{I_{CHG}}{400} R_{ISET} \]

\subsection{System enable input}
Connect SYSOFF high to turn off the FET connecting the battery to the system output. When an adapter is connected, charging is also disabled. Connect SYSOFF low for normal operation. SYSOFF is internally pulled up to VBAT through a large resistor (approximately 5 M$\Omega$). Do not leave SYSOFF unconnected to ensure proper operation.

\subsection{Calculations}
\subsubsection{Program the Fast Charge Current $I_{CHG}$}
\[ R_{ISET} = \frac{K_{ISET}}{I_{CHG}} \]
From the electrical table we can find:
\[ K_{ISET} = 890 A\Omega \]
The charge current limit is adjustable up to 1.5 A. The maximum charge current for the Panasonic NCR18650B battery is 1625 mA. Set the charge current to 1.3 A to have some margin for the battery lifetime.
\[ R_{ILIM} = \frac{890 A\Omega}{1.3 A} = 684.62 \Omega \approx 680 \Omega \]
The valid resistor range is 590 $\Omega$ to 8.9 k$\Omega$. Select the closest standard value, which for this case is 680 $\Omega$. Connect this resistor between ISET (pin 16) and VSS.
\[ I_{CHG} = \frac{890 A\Omega}{680 \Omega} = 1.31 A \]

\subsubsection{Program the Input Current Limit $I_{LIM}$}
\[ R_{ILIM} = \frac{K_{ILIM}}{I_{IN(MAX)}} \]
From the electrical table we can find:
\[ K_{ILIM} = 1610 A\Omega \]
The input current limit is adjustable up to 1.5 A. Set the input current limit higher than the charge current of 1.3 A.
\[ R_{ILIM} = \frac{1610 A\Omega}{1.5 A} = 1.073 k\Omega \approx 1.1 k\Omega  \]
The valid resistor range is 1.1 k$\Omega$ to 8 k$\Omega$. Select the closest standard value, which for this case is 1.1 k$\Omega$. Connect this resistor between ILIM (pin 12) and VSS.
\[ I_{ILIM} = \frac{1610 A\Omega}{1.1 k\Omega} = 1.46 A \]

\subsubsection{Fast-Charge Safety Timer (TMR)}
Leave TMR open to set to default safety timers. Connect to VSS to disable safety timers.

\subsubsection{TS function}
Connect a 10 k$\Omega$ resistor from TS to VSS to set the TS voltage at a valid level and maintain charging.

\subsubsection{Selecting IN, OUT, and BAT Pin Capacitors}
In most applications, all that is needed is a high-frequency decoupling capacitor (ceramic) on the power pin, input, output and battery pins. Using the values shown on the application diagram, is recommended. After evaluation of these voltage signals with real system operational conditions, one can determine if capacitance values can be adjusted toward the minimum recommended values (DC load application) or higher values for fast high amplitude pulsed load applications. Note if designed high input voltage sources (bad adaptors or wrong adaptors), the capacitor needs to be rated appropriately. Ceramic capacitors are tested to 2x their rated values so a 16-V capacitor may be adequate for a 30-V transient (verify tested rating with capacitor manufacturer).

\subsubsection{Selecting MOSFETs}
Discharge Overcurrent Detection (OCD) voltage = 100 mV. $R_{DSON} = 30 m\Omega @ V_{GS} = 4.5 V$
Maximum operating discharge current = $\frac{100 mV}{2*30 m\Omega} = 1.67 A$

\section{Wireless connectivity}
\subsection{Wi-Fi connection}
In order to transmit the sensor data to the server, the Wi-Fi connectivity of the ESP32 is used. The transmission is provided by a TCP connection that is established between the ESP32 and the python server. At the initialisation of the python server, a socket is opened, this socket is used to receive the data. In this manner, the server is always listening on the socket, until a message is received. The ESP32 sends the data in the following order: pixeldata (64 floats), temperature (1 float), audio level (1 float), $CO_2$-level (1 unint16) and TVOC-level (1 unint64). The received data is processed by order to determine which data is represented by the received values. The data is parsed into different variable, these variables are then used to update the values in the database. 

\subsection{Power consumption}
To determine the power that is consumed when the sensor data is transmitted once, a measurement is performed. The current consumption is measured, the result is displayed in figure~\ref{fig:wifi_pwr}.
\begin{figure}[!ht]
	\centering
	\includegraphics[width=1.0\linewidth]{wifi_pwr.pdf}
	\caption{Power consumption of one Wi-Fi transmission}
	\label{fig:wifi_pwr}
\end{figure}
In order to interpret the current consumption, the used capacity is displayed in [mAh]. This results in 0.041 mAh of capacity that is used in order to transmit the sensor data once. With this knowledge, we can calculate the amount of Wi-Fi transmissions that are possible with the 6400 mAh battery:

\begin{gather*}
	\frac{6400}{0.041} \approx 156097
\end{gather*}

156097 transmissions are possible, if there are 6 transmissions per hour and the system is active for 7 hours a day, this leads to a battery life off 3716 days.  However, this calculation only accounts for the Wi-Fi transmission other power consumption such as stand-by power, CPU power and sensor power are not accounted for in the calculation. 

\section{Software design}
\textbf{todo}
\begin{figure}[!ht]
	\centering
	\includegraphics[width=0.8\linewidth]{statendiagram.pdf}
	\caption{State diagram}
	\label{fig:statediagram}
\end{figure}

\section{Energy management - autonomy}
Berekeningen voor levensduur komen hier.

\section{Website}

\section{Financial estimate}
Kostprijs van het project opstellen.

\section{Validation and measurements}
Hier moeten we aantonen of dat alles werkt. Of eventuele fouten verklaren.

\end{document}